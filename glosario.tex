%% ---------------------------------------------------------------------------
%% paNotation.tex
%%
%% Notation
%%
%% $Id: paNotation.tex,v 1.15 2004/03/30 05:55:59 alvarado Exp $
%% ---------------------------------------------------------------------------

\cleardoublepage
\renewcommand{\nomname}{Lista de símbolos y abreviaciones}
\markboth{\nomname}{\nomname}
\renewcommand{\nompreamble}{\addcontentsline{toc}{chapter}{\nomname}%
\setlength{\nomitemsep}{-\parsep}
\setlength{\itemsep}{10ex}
}

%%
% Símbolos en la notación general
% (es posible poner la declaración en el texto
%%

\symg[FFT]{Transformada R\'apida de Fourier}
\symg[SSD]{Suma de Diferencias Cuadradas}
\symg[DNLM]{Filtro Deceived Non Local Means}
\symg[DNLM-MOAS]{Filtro Deceived Non Local Means con Media Movil y Simetr\'ia}
\symg[DNLM-IIFFT]{Filtro Deceived Non Local Means con Im\'agenes Integrales y FFT}
\symg[CPU]{Unidad Central de Procesamiento}
\symg[GPU]{Unidad Gr\'afica de Procesamiento}
\symg[USM]{Filtro Unsharp Mask}



\symg[C]{$\setC$}{Conjunto de los números complejos.}

%%
% Algunas abreviaciones
%%

\syma{PCA}{Análisis de componentes principales}
\syma{WSN}{Redes Inalámbricas de Sensores}
\syma{ASM}{Modelos Activos de Forma}

\printnomenclature[20mm]

%%% Local Variables:
%%% mode: latex
%%% TeX-master: "paMain"
%%% End:
