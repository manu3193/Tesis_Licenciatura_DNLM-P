%% ---------------------------------------------------------------------------
%% paNotation.tex
%%
%% Notation
%%
%% $Id: paNotation.tex,v 1.15 2004/03/30 05:55:59 alvarado Exp $
%% ---------------------------------------------------------------------------

\cleardoublepage
\renewcommand{\nomname}{Lista de símbolos y abreviaciones}
\markboth{\nomname}{\nomname}
\renewcommand{\nompreamble}{\addcontentsline{toc}{chapter}{\nomname}%
\setlength{\nomitemsep}{-\parsep}
\setlength{\itemsep}{10ex}
}

%%
% Símbolos en la notación general
% (es posible poner la declaración en el texto
%%



%%
% Algunas abreviaciones
%%

\syma{FFT}{Transformada R\'apida de Fourier}
\syma{DNLM-MOAS}{Filtro Deceived Non Local Means con Media Movil y Simetr\'ia} 
\syma{DNLM-IIFFT}{Filtro Deceived Non Local Means con Im\'agenes Integrales y FFT} 
\syma{CPU}{Unidad Central de Procesamiento} 
\syma{VPU}{Unidad de Procesamiento Vectorial} 
\syma{USM}{Filtro \engl{Unsharp Mask}} 
\syma{MCDRAM}{Memoria de Alto Ancho de Banda}

\printnomenclature[20mm]

%%% Local Variables:
%%% mode: latex
%%% TeX-master: "paMain"
%%% End:
