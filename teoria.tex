\chapter{Marco teórico}
\label{ch:marco}

%\section{Descripción}
%
%Toda tesis hace referencia a trabajos previos en el área y trabajos afines que
%están directamente relacionados con lo planteado en el tesis.
%
%Además, en el marco teórico debe aparecer la información absolutamente
%necesaria para comprender la solución, y por eso es recomendable escribir
%primero la solución (el siguiente capítulo), para ir anotando qué debe ser
%explicado en el marco teórico.
%
%\section{Generalidades}
%
%Se recomienda revisar las guías de publicación de la \nt{IEEE} en
%\url{http://www.ieee.org/publications_standards/publications/authors/authors_journals.html},
%donde puede encontrar cómo hacer referencias bibliográficas correctamente, cómo
%citar ecuaciones, tablas y figuras, etc.  
%
%\subsection{Redacción}
%
%La \nt{redacción} en todo el documento debe seguir un estilo científico
%objetivo. Esto implica que se redacta de modo impersonal, sin utilizar primeras
%personas del singular o del plural, y se evita el uso de cualquier tipo de
%calificativo, sustituyéndolos siempre por datos concretos, vinculados a
%referencias bibliográficas o datos experimentales. Los comparativos también
%deben concretarse a hechos y datos, y nunca dejarse ``en el aire''. Por la
%naturaleza de la tesis, el tiempo verbal es usualmente presente, no perdiendo
%nunca de vista que se está explicando ``cómo hacer algo'', en vez de ``qué se
%hizo''.
%
%Las \nt{frases} deben ser cortas, y debe evitarse que el lector tenga que saltar
%constantemente entre partes de la tesis, lo que implica una exposición lineal
%clara, donde lo que se necesita ya ha sido explicado antes. Deben evitarse
%redundancias y por tanto cada concepto se exponen en un único lugar.
%
%Todo aspecto circunstancial es irrelevante para la tesis, es decir, si se ha
%desarrollado en el laboratorio $X$, o en el curso $Y$, con el profesor $Z$, o
%en la empresa $W$, el nombre de funciones o clases en su código, etc., es
%información irrelevante para reproducir el experimento, y por lo tanto sobra.
%
%\subsubsection{Numeración del documento}
%
%La primera página de la tesis es la correspondiente a la introducción,
%así que ésta debe ser la página 1. Desde la introducción, hasta antes
%de la bibliografía, las unidades son ``Capítulos''. La bibliografía y
%anexos no se consideran capítulos, así que ya no continúan con la
%misma numeración de los capítulos (la paginación sí continua). Los
%índices, notación, glosario, etc.\ se numeran con números romanos en
%versalitas ({\textsc{I}, \textsc{II}, \textsc{III}, \textsc{IV},
%  \textsc{V}, \textsc{VI}}$\ldots$) y antes del índice (portada,
%resúmenes, agradecimientos, hoja de evaluadores, etc.) las páginas no
%llevan numeración.
%
%Esta plantilla LaTeX ya se ocupa de todo lo anterior.
%
%\subsection{Ecuaciones}
%
%Para citar \nt{ecuaciones} se utilizan paréntesis redondos, y no es
%necesario emplear explícitamente la palabra ``ecuación''. Por ejemplo
%``Introduciendo en (4.2) los resultados de (3.3) y (3.7) se obtiene
%...''. La ecuación es parte del flujo de texto y no un objeto
%flotante, así que no pueden emplearse como figuras. Cuando se requiere
%la ecuación, allí se inserta.
%
%Es incorrecto redactar de la siguiente forma: \explain{MAL}
%
%\textsl{La operación del transistor sin tomar en cuenta el efecto Early está
%  dada por (\ref{eq:ej1}), donde el parámetro $\kappa$ está dado por
%  (\ref{eq:ej2}).}
%
%\begin{equation} \label{eq:ej1}
%  I_{DS}
%  =
%  I_{n0} \frac{W}{L}e^{\kappa \frac{V_{GB}}{v_t}}
%  \left[
%    e^{-\frac{V_{SB}}{v_t}}
%    -
%    e^{-\frac{V_{DB}}{v_t}}
%  \right]
%\end{equation}
%
%\begin{equation} \label{eq:ej2}
%  \kappa = \frac{C_{ox}}{C_{ox}+C_{dep}}
%\end{equation}
%
%Lo anterior es incorrecto porque obliga al lector a estar buscando ecuaciones,
%que pueden mostrarse directamente.  La única referenciación permitida es hacia
%atrás.
%
%La forma correcta de redactar lo anterior es: \chk{BIEN}
%
%\textsl{La operación del transistor sin tomar en cuenta el efecto Early está
%  dada por}
%\begin{equation} \label{eq:ej3}
%  I_{DS}
%  =
%  I_{n0} \frac{W}{L}e^{\kappa \frac{V_{GB}}{v_t}}
%  \left[
%    e^{-\frac{V_{SB}}{v_t}}
%    -
%    e^{-\frac{V_{DB}}{v_t}}
%  \right]
%\end{equation}
%\textsl{donde el parámetro $\kappa$ es}
%\begin{equation} \label{eq:ej4}
%  \kappa = \frac{C_{ox}}{C_{ox}+C_{dep}}
%\end{equation}
%
%Así el flujo del texto guía al lector por las ecuaciones sin mayor esfuerzo.
%
%Es recomendable numerar \emph{todas} las ecuaciones, de modo que en la revisión
%del documento, o en futuras referencias a su documento de tesis todas las
%ecuaciones puedan ser citadas sin requerir describir textualmente a cuál
%ecuación se está haciendo referencia.
%
%\subsection{Figuras}
%
%Para el almacenamiento de imágenes existen dos tipos de formato: las imágenes
%raster y las imágenes vectoriales.\index{imagen!raster}
%
%\subsubsection{Imágenes raster}
%
%Las imágenes raster son representadas por una rejilla de píxeles, en donde cada
%píxel tiene un valor que representa al nivel de gris o el color. La
%discretización espacial es ineludible, y la única forma de obtener buena
%calidad es empleando tamaños grandes de la imagen que conduzcan a resoluciones
%de al menos 300 puntos por pulgada en la impresión, lo que conlleva a archivos
%de documentos de varios megabytes. Dentro de los formatos para almacenar
%imágenes raster existen algunos con pérdida (como el JPEG) que producen en
%imágenes sintéticas, como diagramas, estructuras ruidosas que dan una
%apariencia de baja calidad a las figuras. Otros formatos (como PNG, BMP, TIFF o
%GIF) no tiene pérdidas de información, pero los algoritmos de compresión no
%pueden reducir el tamaño de las imágenes con los mismos factores de reducción
%que los formatos con pérdidas. Este tipo de formatos debe utilizarse únicamente
%para fotografías o capturas de escenas reales con cámaras digitales.
%
%\subsubsection{Imágenes vectoriales}
%
%\index{imagen!vectorial}
%Las imágenes vectoriales \textbf{deben} ser empleadas en todo tipo de
%diagrama. En ellas no se almacenan píxeles, sino las estructuras geométricas
%que componen la figura como círculos (representado por posicion de su centro y
%su radio), rectángulos (representados por sus esquinas), líneas, texto, etc. La
%mayoría de programas para elaborar este tipo de diagramas, como Inkscape, XFig,
%OpenOffice.org Draw, MS Visio, Adobe Illustrator, etc. proveen varios formatos
%vectoriales que pueden ser insertados tanto en LaTeX como en OpenOffice.org
%Writer (o MS Word). Los formatos más empleados son los llamados metafiles, que
%incluyen al WMF, EMF. En LaTeX se utiliza por lo general EPS. Recientemente se
%ha incrementado el soporte al formato SVG.
%
%No debe cometerse el error de generar una imagen vectorial a partir de una
%imagen raster, pues una vez realizada la discretización espacial no es posible
%reconstruir los elementos geométricos que componen la imagen. Por ello, no
%tiene ningún sentido generar un archivo EPS o WMF a partir de una imagen ya
%almacenada en BMP, JPG, o PNG, pues lo único que ocurrirá es que se inserta la
%figura raster tal cual en la imagen vectorial, sin implicar ninguna ganancia en
%la calidad.
%
%Esta plantilla de LaTeX administra la generación de ciertas figuras por usted.
%Puede colocar en el directorio \texttt{fig/} archivos EPS, JPG, PNG o GP (de
%GNUPlot) y el Makefile se encarga de hacer todas las conversiones necesarias.
%En las siguientes subsecciones se describen dos casos adicionales que resultan
%útiles para realizar figuras más complejas.
%
%\subsubsection{Figuras ltxfig/psfrag}
%
%\index{psfrag}\index{ltxfig}
%Cuando en el subdirectorio \texttt{fig/} se encuentran dos archivos con el
%mismo nombre pero extensiones \texttt{ltxfig} y \texttt{psfrag}, por ejemplo
%\texttt{prueba.ltxfig} y \texttt{prueba.psfrag}, entonces el Makefile asume que
%usted desea crear una figura a partir del archivo \texttt{prueba.ltxfig},
%creado con el programa \texttt{XFig}, sustituyendo los textos ahí presentes con
%texto formateado con LaTeX.
%
%La figura~\ref{fig:ltxfig} ha sido creada con este esquema.  Revise los
%archivos correspondientes en el directorio de figuras
%\texttt{fig/ltxfig\_prototipo.*} para más detalles sobre su uso.
%
%\begin{figure}[htb]
%  \centering
%  \includegraphics[width=0.9\textwidth]{ltxfig_prototipo}
%  \caption{Ejemplo de imagen ltxfig/psfrag}
%  \label{fig:ltxfig}
%\end{figure}
%
%\subsubsection{Figuras pstricks}  
%
%\index{pstricks}
%Los archivos con extensión \texttt{.pstricks} en el directorio \texttt{fig} se
%utilizan para generar cualquier tipo de imágenes según el código que se
%contenga.  Es un concepto más general que el anterior.  La
%figura~\ref{fig:pstricks} ha sido creada con este esquema.  Puede revisar los
%archivos \texttt{prototipo\_gnuplot*} como un ejemplo de su uso, en donde de un
%archivo gnuplot (\texttt{\_.gp}) se genera un archivo \texttt{\_.eps}, el cual
%es incluido en el archivo \texttt{.pstricks} sustituyendo cadenas de texto por
%código LaTeX.
%
%\begin{figure}[htb]
%  \centering
%  \includegraphics{prototipo_gnuplot}
%  \caption{Ejemplo de imagen gnuplot/pstricks}
%  \label{fig:pstricks}
%\end{figure}
%
%\subsubsection{Entradas en el índice de figuras}
%
%El índice de figuras debe servir para encontrar rápidamente dónde se
%encuentra cierta figura.  El pie de la figura, indicado en \LaTeX con
%\texttt{caption} puede ser extenso, en especial para indicar detalles
%de las figura, y es la entrada por defecto que aparecerá en el índice
%de figuras, la cual no debe superar la extensión de una línea y debe
%únicamente dar la idea del contenido de la figura para poder ser
%encontrada.  Para lograr esto en \LaTeX{} se agrega un parámetro
%opcional con el texto del índice de la siguiente forma:
%\begin{verbatim}
%  \caption[Texto en el índice]{Texto al pie de la figura}
%\end{verbatim}
%
%\subsection{Referencias bibliográficas}
%
%\index{referencias}\index{BibTeX}
%Todo concepto o idea tomado de otros autores contar con la respectiva
%referencia. En redacción técnica de ingeniería rara vez se utiliza la cita
%textual, así que es necesario reformular las ideas y conceptos con palabras
%propias. En ingeniería electrónica se utilizan los formatos de referencia de la
%IEEE o la ACM, que son numéricos, encerrados entre paréntesis cuadrados (por
%ejemplo, ``En \cite{Davis1963} se propuso un nuevo algoritmo'', o ``En
%\cite{ProakisManolakis1998} los autores proponen tomar las ventajas de los
%algorimos presentados en \cite{Oppenheim1998,Roberts2005,Haykin2001} por medio
%del método de Newton \cite{Burrus1998} conocido en el área de optimización
%lineal.''). La referencia es parte de las frases, así que si la frase termina
%con la referencia para indicar la idea, ésta debe estar antes del punto final o
%demás signos de puntuación: ``La capacidad de memoria también sigue una Ley
%similar a la de Moore \cite{Octave}. Los siguientes son los aspectos a tomar en
%cuenta en el diseño del sistema \cite{Lindner2002}:''
%
%Se recomienda utilizar BibTeX para indicar las referencias
%bibliográficas.  Actualmente herramientas como Mendeley, Zotero u
%otras similares simplifican la administración de las referencias y
%pueden exportar al formato BibTeX.
%
%\subsection{Extensión}
%
%\index{extensión}
%Una tesis de licenciatura no debe sobrepasar las 120 páginas incluyendo
%apéndices y los formalismos desde portada hasta índices.
%
%El cuerpo de la tesis (desde introducción hasta conclusiones) usualmente se
%extiende desde 45 páginas hasta no más de 80, dependiendo de la problemática
%tratada.
%
%No es necesario reproducir contenidos de otras fuentes: agregue las referencias
%a dichas fuentes, y limítese a enunciar lo estrictamente necesario para
%comprender sus propuestas de solución.
%
%\section{Sobre esta plantilla \LaTeX}
%
%Esta plantilla \LaTeX pretende simplificar varios pasos en la creación del
%documento de tesis.
%
%\subsection{Marcar asuntos pendientes}
%
%La plantilla tiene dos ``\emph{modos}'' de operación: normal y borrador
%(\emph{draft}).  En el archivo \texttt{main.tex} a partir de la línea 41 usted
%encuentra el código
%
%\begin{verbatim}
%%
%% DRAFT MODE
%%
%\newboolean{draftmode}                  % boolean used to control draft-mode
%% Ensure that only one of the next two lines is active:
%\setboolean{draftmode}{true}            % turn draft mode on
%%\setboolean{draftmode}{false}           % turn draft mode off
%\end{verbatim}
%
%Con el modo borrador, se activan ciertos comandos y funcionalidades útiles en
%el proceso de elaboración de la tesis, pero que deben ser desactivados al
%final, antes de entregar la tesis.  Por ejemplo, se activa el pie de página que
%dice ``\emph{Borrador: fecha}'', y se activa el índice titulado ``Revisar''.  En dicho índice aparecen las páginas en donde se hayan utilizado alguno de los siguientes comandos:
%\begin{compactitem}
%\item \verb+\boxcomment{comentario}+ Crea una caja en el margen de página con
%  el comentario indicado.
%\item \verb+\explain{comentario}+ Crea una caja en el margen de página con
%  el comentario indicado, con una flecha hacia la derecha para indicar qué en
%  concreto debe ser revisado.
%\item \verb+\chk{comentario}+ Crea una caja en el margen con símbolo de
%  ``chequeado'' y el comentario indicado.
%\item \verb+\TODO{comentario}+ Crea una caja grande de fondo sombreado con el
%  comentario indicado.
%\end{compactitem}
%
%En este párrafo se\chk{resultado de chk} utilizan algunos de estos comandos
%para ilustrar su efecto.  El \verb+\chk+ como puede observar tiene sentido
%usarlo para marcar que algo está casi listo.  Por otro lado \explain{explain}
%el comando \verb+\explain+ permite marcar algo que requiere ser revisado en
%redacción, valores, etc.  El \verb+\boxcomment+\boxcomment{La caja simple}
%solo pone una marca al margen.
%
%\TODO{Finalmente el comando \texttt{TODO} coloca esta caja gris.}
%
%Si usted desativa el modo draft, desaparecen todas las marcas
%anteriores, y desaparece el índice ``Revisar''.  En éste índice
%aparecen todas las páginas en donde se utilizaron estos comandos con
%los respectivos comentarios, lo que permite encontrar rápidamente
%detalles que usted indicó que debe revisar.
%
%\subsection{Índices}
%
%Como índice se conoce la lista de términos claves con su respectiva
%página.  Usualmente aparece al final del documento.  La plantilla
%ofrece varios comandos para simplificar el uso estandar del comando de
%\LaTeX\ \verb+\index{termino}+ que coloca al término indicado en el
%índice.  Con \verb+\nt[indice]{termino}+ (\emph{new term}) usted
%indica la entrada principal del término, que aparece en el texto en el
%índice, es decir, en el índice aparece lo que indique en vez de
%``indice'' y en el texto aparece lo que indique ``termino'';
%\verb+\ot{termino}+ agrega una entrada secundaria al término.

En este cap\'itulo se detalla el contenido te\'orico necesario para el desarrollo de la propuesta presentada en este trabajo. La propuesta consiste en la paralelizaci\'on del filtro Deceived Non-Local Means (DNLM), presentado como parte de la familia de filtros DeWAFF en \cite{calderon2015dewaff}, para la eliminaci\'on de ruido, realce de bordes y mejora de contraste en la imagen. La propuesta de paralelizaci\'on incluye la evaluaci\'on de optimizaciones computacionales desarrolladas para este filtro y para el filtro Non-Local Means original (NLM) presentadas en \cite{CalderonRamirez2017} y \cite{Condat2010, Darbon2008}, respectivamente.
Como lo demuestra el autor en \cite{calderon2015dewaff}, el filtro DNLM presenta los mejores resultados en comparaci\'on con el filtro bilateral y el filtro bilateral escalado (ambos parte de DeWAFF), ya que se basa en el efoque de eliminaci\'on de ruido por medio del ponderado no local presentado en \cite{buades2005non}. Sin embargo, la complejidad computacional del filtro DNLM hace que su uso sea poco pr\'actico, debido a que los investigadores frecuentemente necesitan analizar videos de actividad celular compuestos por cientos de miles de im\'agenes  \cite{Yang2006NucleiSU, cellsegmentationMarkov, Tay2010, Fils_BfilCells_2008}.





de im\'agenes que combina la eliminaci\'on de ruido y la mejora de informaci\'on importante como los bordes y el contraste. La reducci\'on de ruido se realiza por medio del enfoque de filtrado espacial por ponderado no local, llamado filtro Non-Local Means. Este enfoque presenta buena robustez ante el ruido, pero una mayor complejidad computacional, la cual se aborda con mayor detalle en este cap\'itulo. La mejora en la imagen se realiza utilizando uno de los enfoques de Unsharp Masking, permitiendo un mayor realce en bordes y contraste en la imagen. Adem\'as se presenta una estrategia de paralelizaci\'on que permita el enfoque del desacople entre la imagen de filtrado y la utilizada para el pesado, necesario para la versi\'on mejorada presentada en \cite{deceived}., con el fin de reducir el tiempo de procesamiento del algoritmo en una primera instancia y evitar los artefactos que resultan al utilizar el enfoque tradicional de filtrado en etapas sucesivas. 


\subsection{Conceptos B\'asicos}

\subsubsection{Filtrado Espacial}

Los m\'etodos de filtrado presentados en este cap\'itulo se enfocan en el dominio espacial. La imagen es representada como una matriz que contiene los valores de los pixeles en un plano de coordenadas (x,y). Este plano representa el dominio espacial, en el que se realizan operaciones directamente en la matriz de pixeles, a diferencia de otros m\'etodos basados en transformaciones (transformada de Fourier, Wavelets o histogramas, por dar un ejemplo).



\subsubsection{Filtros lineales y filtros no lineales}


Los filtros lineales se basan en la convoluci\'on de kernels con la se\~nal correspondiente a los pixeles en la imagen. Estos filtros comparten los principios de surpeposici\'on y homogeneidad y se emplean para efectos de suavizados o agudizamiento de la imagen. Existen distintos tipos de filtros lineales, incluyendo  Gaussianos, pasa baja y pasa alta, entre otros. 


En una imagen los cambios de alta magnitud entre el valor de pixeles se asocian a altas frecuencias, como es el caso de ruido y bordes. Los cambios suaves corresponden a bajas frecuencias, por lo tanto los filtros pasa baja atenuan las altas frecuencias provocando una disminuci\'on entre cambios bruscos y por lo tanto eliminaci\'on de ruido, pero con el costo de afectar detalles y bordes.

Otro ejemplo de filtros lineales son los Gaussianos que permiten un efecto equivalente para un pixel en todas las direcciones, dando un mejor efecto de suavizado. Tambi\'en los filtros pasa alta se utilizan para la detecci\'on de bordes y

Los filtros no lineales permiten generar modificaciones m\'as complejas a la imagen y evitar algunos problemas presentes en los filtros lineales como lo son la degradaci\'on de los bordes si se habla de los filtros pasa baja, o en el caso de los filtros pasa alta, la magnificaci\'on del ruido.

\begin{equation}
\label{eq:mediana}
\Gamma_{mediana}=\begin{array}
\left\Vert \begin{array}{ccc}
1 & 1 & 1\\
1 & 1 & 1\\
1 & 1 & 1
\right\Vert\end{array}
\end{array}
\end{array}\right\Vert
 \enspace .
\end{equation}

El filtro de mediana es un ejemplo de filtro no lineal su kernel se encuentra definido por el kernel \ref{eq:mediana}. 




\subsection{Realce de detalles en imagen}\

\subsubsection{Unsharp Masking}

Para el realce de los bordes y mejora de contraste se utilza uno de los m\'etodos de unsharp mask que realiza una substracci\'on de la imagen suavizada a la imagen original dando como resultado una imagen $B$ con las altas frecuencias de la imagen, es decir, los bordes y otros cambios considerables en la intensidad de los pixeles t\'ipicamente representados como detalles o ruido.

\begin{equation}
\label{eq:unsharpmask}
G=U+\lambda\,B
\end{equation}

La imagen $G$ corresponde a la imagen mejorada resultante al proceso de Unsharp Masking, en este caso, el coeficiente $\lambda$ regula el grado de mejora. 


\subsection{Eliminaci\'on de ruido}

\subsubsection{Ruido Gaussiano Aditivo}

\subsubsection{Filtro Non-Local Means}

El filtro de promediado no local o Non-Local mean utiliza la distancia euclidiana para determinar la similitud entre vecindarios de dos pixeles $p_{1}=\left(x_{1},y_{1}\right)$
y $p_{2}=\left(x_{2},yx_{2}\right)$ y as\'i determinar el pondeado en una regi\'on dada por una ventana $\omega$ para asignarle el peso correspondiente al pixel reduciendo los cambios bruscos a la vez que se conservan los bordes y detalles, tomando como punto de partida que la localidad est\'a muchas veces relacionada con la similitud de los patrones. La ecuaci\'on del filtro NLM se expresa de la siguiente manera:

\begin{equation}
\label{eq:nlmfunc}
\[
\psi_{\textrm{NLM}}\left[\left[u,v\right],\left[x,y\right]\right]=\exp\left(-\frac{\left\Vert \vec{\eta}\left[x,y\right]-\vec{\eta}\left[u,v\right]\right\Vert }{2\sigma_{r}^{2}}\right).
\]
\end{equation}

\subsubsection{Filtro Deceived Non-Local Means}

El filtro Deceived Non-Local Means (DNLM) presenta un enfoque novedoso el cual consiste en la combinaci\'on de un m\'etodo de Undsharp Masking (USM) con el filtro Non Local Means, con el fin de lograr una mejora en los bordes y en el contraste de la imagaen, as\'i como la eliminaci\'on de ruido aditivo gausiano  presente en la imagen. La combinaci\'on propuesta se basa en el desacople entre la imagen para el pesado (imagen original $U$) y la utilizada en el filtrado $U_{USM}$. Esto permite evitar el efecto anillo presente en las im\'agenes filtradas con el USM. La siguientes ecuaciones presentan la definici\'on del enfoque de desacoplamiento y la funci\'on de pesado del filtro DNLM respectivamente:

\begin{equation}
\ref{eq:dnlm}
Y(p)=\left(\sum_{m\in \Omega}\psi_{i}\left(U, p, m\right)\right)^{-1} \\ \left(\sum_{m\in \Omega}\psi_{i}\left(U, p, m\right)U_{USM}(m)\right),
\end{equation}


\begin{equation}
	\psi_{\text{NLM}}(p,m) =  exp\left(-\frac{ \left\| \displaystyle\ \overrightarrow{\eta}(\omega_{a},m) - \overrightarrow{\eta}(\omega_{a},p)  \right\|^{2}}{h}\right) 
\end{equation}


En cuanto a la complejidad computacional del filtro, para el peor escenario de usar una ventana $|\Omega| = N$ y un vecindario de tama\~no $\omega_{a} \times \omega_{a} = N$ pixeles, la complejidad computacional del filtro DNLM es de $O(N^3)$, donde $N$ corresponde a el n\'umero total de pixeles de la imagen de entrada $U$. 

\subsection{Optimizaciones computacionales}

\subsection{Paralelizaci\'on}

\subsection{Arquitectura Intel Xeon Phi Knights Landing}