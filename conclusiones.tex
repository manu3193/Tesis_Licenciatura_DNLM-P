\chapter{Conclusiones}
\label{ch:concl}

Los resultados obtenidos evidencian que el enfoque de paralelizaci\'on del filtro DNLM-MOAS-P presenta una aceleraci\'on de hasta $665.63X$, muy cercano a las aceleraciones obtenidas para implementaciones en GPU de hasta $740X$. Con esto se demuestra que se pueden obtener aceleraciones importantes sin tener que emplear lenguajes de programaci\'on paralela m\'as complejos y de un m\'as bajo nivel como CUDA o OpenCL.


En comparaci\'on con las otras implementaciones paralelas para la plataforma Xeon Phi presentes en la literatura, este trabajo alcanza una aceleraci\'on de m\'as de 6 veces. Esto es probablemente porque los autores utilizan la versi\'on anterior del Xeon Phi utilizada en este trabajo, y adem\'as no hacen menci\'on de optimizaciones algor\'itmicas del filtro ni tampoco sobre la vectorizaci\'on, lo cual influye en gran medida, seg\'un se puede comprobar en los resultados de este trabajo. Sin embargo, el alto ancho de banda de memoria que proporciona la arquitectura Xeon Phi KNL permite al filtro DNLM-MOAS-P obtener el menor resultado al escalar a 32 hilos de ejecuci\'on.

Al poner las optimizaciones del filtro DNLM en contraste, los resultados demuestran que el filtro DNLM-IIFFT-P tiene buena escalabilidad al aumentar los hilos de ejecuci\'on, a diferencia del filtro DNLM-MOAS-P que su escalabilidad ve reducida por el ancho de banda. 


El tiempo de filtrado para una imagen de entrada de $1024x1024$ pixeles disminuye de 163s a 0.25s. Como se mencion\'o en un inicio, investigadores de microbiolog\'ia de la Universidad de Costa Rica necesitan analizar $170000$ im\'agenes de actividad celular. Esto significa que con el filtro DNLM-MOAS-P el tiempo de procesamiento de estas im\'agenes disminuya de meses a unas 11.80 horas para im\'agenes de $1024x1024$. Este tiempo puede disminuirse a\'un m\'as si se aumenta la escalabilidad horizontal a procesamiento multi-nodo, lo cual es parte del trabajo futuro que se pretende explorar con m\'etodos como MPI.

