\chapter{Conclusiones}
\label{ch:concl}

Los resultados obtenidos evidencian que el enfoque de paralelización del filtro DNLM-MA-P presenta una aceleración de hasta $665.63\times$, muy cercano a las aceleraciones obtenidas para implementaciones en GPU de hasta $740\times$. Con esto se demuestra que se pueden obtener aceleraciones importantes sin tener que emplear lenguajes de programación paralela más complejos y de un más bajo nivel como CUDA o OpenCL.


En comparación con las otras implementaciones paralelas para la plataforma Xeon Phi presentes en la literatura, este trabajo alcanza una aceleración de más de 6 veces. Esto es probablemente porque los autores utilizan la versión anterior del Xeon Phi utilizada en este trabajo, y además no hacen mención de optimizaciones algorítmicas del filtro ni tampoco sobre la vectorización, lo cual influye en gran medida, según se puede comprobar en los resultados de este trabajo. Sin embargo, el alto ancho de banda de memoria que proporciona la arquitectura Xeon Phi KNL permite al filtro DNLM-MA-P obtener el menor resultado al escalar a 32 hilos de ejecución.

Al poner las optimizaciones del filtro DNLM en contraste, los resultados demuestran que el filtro DNLM-IIFFT-P tiene buena escalabilidad al aumentar los hilos de ejecución, a diferencia del filtro DNLM-MA-P que su escalabilidad ve reducida por el ancho de banda. 

El paralelismo a nivel de datos se empleó con resultados mayores al 81\% de vectorización para el filtro DNLM-IIFFT, 95\% de vectorización en el filtro DNLM y por último 100\% de vectorización para el filtro DNLM-MA. Como se menciona anteriormente, el grado de vectorización alcanzado en este trabajo es clave para lograr altas aceleraciones como las reportadas en la sección \ref{ch:res}.

La paralelización a nivel de tareas demostró comportarse distinto para las diferentes versiones del filtro DNLM. Limitaciones como el ancho de banda en el caso del DNLM-MA-P no permite su escalamiento con un número de hilos mayor a 32. En ese sentido, el filtro DNLM-IIFFT-P y DNLM-P presentan mejor escalabilidad pero con tiempos de procesamiento mayores que el DNLM-MA-P.

El tiempo de filtrado para una imagen de entrada de $1024\times1024$ pixeles disminuye de $163$ s a $0,25$ s. Como se mencionó en un inicio, investigadores de microbiología de la Universidad de Costa Rica necesitan analizar $170000$ imágenes de actividad celular. Esto significa que con el filtro DNLM-MA-P el tiempo de procesamiento de estas imágenes disminuya de meses a unas $11,80$ horas para imágenes de $1024\times1024$. Este tiempo puede disminuirse aún más si se aumenta la escalabilidad horizontal a procesamiento multi-nodo, lo cual es parte del trabajo futuro que se pretende explorar con métodos como MPI.

